\documentclass{article}
\title{Guia de ejercicos geometria - Triangulos 1}
\date{23-11-2022}
%\author{Santiago I. Flores}

\usepackage{amsmath}
% \usepackage[spanish]{babel}
\usepackage[utf8]{inputenc}
\usepackage{multicol}
\setlength{\columnsep}{1cm}

% Definiendo tipo de letra
%\usepackage{fontspec}
%\defaultfeatures{Ligature=TeX}
%\setromanfont{Carlito}
\renewcommand{\familydefault}{\sfdefault}

% Ajuste del tamaño de los margenes
\usepackage[a4paper]{geometry}
\setlength{\columnseprule}{1pt}
\geometry{top=0.10cm, bottom=1.25cm, left=0.5cm, right=0.5cm}

% Quitando la numeración de paginas
\thispagestyle{empty}

\begin{document}
\maketitle

\begin{multicols}{2}

\begin{enummerate}
	
	\item Hallar "$\alpha$"
	\item Hallar "$\alpha$"
	\item Hallar "$x$"
	\item Hallar "$\alpha$"
	\item Hallar "$x$"
	\item Los lados de un tri\'angulo is\'osceles miden 5 y 13. Hallar su per\'imetro.
	\item En la figura, ABC es un tri\'angulo equil\'atero. Calcular "$x$".
	\item Encontrar el valor de "$x$".
	\item Calcular: $x+y+z+w$
	\item Halla la suma de los valores enteros que admiten el valor de "$x$".
	\item Las medidas de los \'angulos agudos de un tri\'angulo rect\'angulo se diferencian en 20

\end{enumerate}

\end{multicols}
\end{document}

